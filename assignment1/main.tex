% \iffalse
\let\negmedspace\undefined
\let\negthickspace\undefined
\documentclass[journal,12pt,twocolumn]{IEEEtran}
\usepackage{cite}
\usepackage{amsmath,amssymb,amsfonts,amsthm}
\usepackage{algorithmic}
\usepackage{graphicx}
\usepackage{textcomp}
\usepackage{xcolor}
\usepackage{txfonts}
\usepackage{listings}
\usepackage{enumitem}
\usepackage{mathtools}
\usepackage{gensymb}
\usepackage{comment}
\usepackage[breaklinks=true]{hyperref}
\usepackage{tkz-euclide} 
\usepackage{listings}
\usepackage{gvv}                                        
\def\inputGnumericTable{}                                 
\usepackage[latin1]{inputenc}                                
\usepackage{color}                                            
\usepackage{array}                                            
\usepackage{longtable}                                       
\usepackage{calc}                                             
\usepackage{multirow}                                         
\usepackage{hhline}                                           
\usepackage{ifthen}                                           
\usepackage{lscape}
\usepackage{multicol}

\newtheorem{theorem}{Theorem}[section]
\newtheorem{problem}{Problem}
\newtheorem{proposition}{Proposition}[section]
\newtheorem{lemma}{Lemma}[section]
\newtheorem{corollary}[theorem]{Corollary}
\newtheorem{example}{Example}[section]
\newtheorem{definition}[problem]{Definition}
\newcommand{\BEQA}{\begin{eqnarray}}
\newcommand{\EEQA}{\end{eqnarray}}
\newcommand{\define}{\stackrel{\triangle}{=}}
\theoremstyle{remark}
\newtheorem{rem}{Remark}
\begin{document}
\parindent 0px

\bibliographystyle{IEEEtran}
\vspace{3cm}

\title{Quadratic Equation and Inequalition(Inequalities)}

\author{AI24BTECH11027 - R Sumanth }
\maketitle 
\newpage
\bigskip

\renewcommand{\thefigure}{\theenumi}
\renewcommand{\thetable}{\theenumi}
\section*{C\quad\quad MCQs with One Correct Answer}
\begin{enumerate}

\item If $\ell,m,n$ are real, $\ell\neq m$, then the roots by the  equation: 
\hfill (1979)
 $\brak{\ell-m}x^2-5\brak{\ell+m}x-2\brak{\ell-m}$
\begin{enumerate}
\item Real and equal\item complex\item Real and unequal\item None of the above 
\end{enumerate}
\item  The equation $x+2y+2z$=1 and 2$x+4y+4z$=9 have \hfill (1979)
\begin{enumerate}
\item Only one solution\item Only two solutions\item Infinite number of solutions\item None of these.
\end{enumerate}
\item  if $x,y$ and $z$ are real and different and 
$u=x^2+4y^2+9z^2-6yz-3zx-2xy$,then u is always.
\hfill (1979)
\begin{enumerate}
\item non negative\item zero\item non  positive\item none of the above
\end{enumerate}
\item let $a>0,b>0$ and $c>0$.Then the roots of  the equation 
$ax^2+bx+c$=0 \hfill (1979)
\begin{enumerate} 
\item are real and negative\item have negative real parts\item both (a) and (b)\item none of the above
\end{enumerate}
\item Both the roots of the equations \\
$\brak{x-b}\brak{x-c}+\brak{x-a}\brak{x-c}+\brak{x-a}\brak{x-b}$ = 0 are always\hfill (1980)
\begin{enumerate} 
\item positive\item real\item negative\item none of these.
\end{enumerate}
\item The least value of the expression $2log_{10} x-log_x\brak{0.01}$,for $x>1$,is \hfill (1980)
\begin{enumerate} 
\item 10\item 2\item -0.01\item none of these.
\end{enumerate}
\item   If$\brak{X^2+px+1}$is a factor of$\brak{ax^3+bx+c}$,then

\hfill (1980)
\begin{enumerate} 
\item$a^2+c^2=-ab$\item $a^2-c^2=-ab$\item $a^2-c^2=ab$\item none of these
\end{enumerate}
\item The no of real solution in equation $|x^2|-3|x|+2=0$ is \hfill (1982 -2 Marks)
\begin{enumerate} 
\item 4\item 1\item 3\item 2
\end{enumerate}
\item Two towns A and B are 60km apart. A school is to built to serve 150 students in a town B. If the total distance to be travelled by all 200 students is to be as small as possible, then the school should be build at \hfill (1982 - 2 Marks)
\begin{enumerate}
\item town B\item 45 km from town A\item town A\item 45 km from town B
\end{enumerate}
\item If $p,q,r$ are any real numbers, then 

\hfill (1982 - 2 Marks)
\begin{enumerate}
\item max$\brak{p,q,r}<$max$\brak{p,q}$\item min$\brak{p,q}$=$\frac{1}{2}$$\brak{p+q}$\item max$\brak{p,q}<$min$\brak{p,q,r}$\item  none of these
\end{enumerate}
\item The largest interval for which $x^{12}-x^9+x^4-x+1>0$ is \hfill (1982 - 2 Marks)
\begin{enumerate}
\item $-4<x\leq0$\item $0<x<1$\item $-100<x<100$\item $-\infty<x<\infty$
\end{enumerate}
\item The equation $x-\frac{2}{x-1}=1-\frac{2}{x-1}$ has

\hfill (1984 - 2 marks)
\begin{enumerate}
\item no root\item one root\item two equal roots\item infinitely many roots 
\end{enumerate}
\item  If $a^2+b^2+c^2=1$, then $ab+bc+ca$ lies in the interval \hfill (1984 - 2 Marks)
\begin{enumerate}
\item {[$\frac{1}{2}$,2]}\item {[-1,2]}\item {[-$\frac{1}{2},1$]}\item {[-1,$\frac{1}{2}$]}
\end{enumerate}
\item   If $log_{0.3}$$\brak{x-1}$$<$$log_{0.09}$$\brak{x-1}$,then $x$ lies in the interval 
\hfill (1985 - 2 Marks)
\begin{enumerate}
\item $\brak{2,\infty}$\item $\brak{1,2}$\item $\brak{-2,-1}$\item none of these
\end{enumerate}
\item If $\alpha$,$\beta$ are the roots of $x^2+px+q=0$ and $\alpha^4,\beta^4$ are the roots of $x^2-rx+s=0$. then the equation $x^2-4qx+2q^2-r=0$ has always
 
\hfill(1989 - 2 Marks)
\begin{enumerate}
\item(a) two real roots\item(b) two positive roots\item(c) two negative roots\item(d) one positive and one negative root
\end{enumerate}
\end{enumerate}
\end{document}

































